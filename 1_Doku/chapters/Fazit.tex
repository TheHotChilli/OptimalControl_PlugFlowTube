\chapter{Fazit}
Dieses Projekt beschäftigte sich mit der Modellierung der Reaktionsgleichung 
\begin{align}
	A_1 \ce{<=>} A_2  \ce{->} A_3
\end{align}
welche innerhalb eines Plug Flow Tube Reactors über eine Steuerung beeinflusst werden konnte. Ziel war es eine Steuerung zu ermitteln, welche die Konzentration des Reaktionsproduktes $A_2$ maximiert. Für diesen Zweck wurden grundsätzlich zwei verschiedene Ansätze untersucht. 
\\Ein Ansatz nutzt direkte Lösungsmethoden, d.h. das Optimalsteuerungsproblem wird im ersten Schritt auf ein endlichdimensionales Optimierungsproblem zurückgeführt. Zu diesem Zweck wird das Differentialgleichungssystem zu Beginn diskretisiert und dann ein Optimierungsverfahren zur Maximierung der Konzentration $x_2(t)$ verwendet. Für die Diskretisierung wurden drei verschiedene Verfahren ausprobiert: Das explizite Euler-, das implizite Radau2A- und das Matlab-Verfahren \textit{ODE23s}. Alle drei Methoden lieferten gute Ergebnisse, wobei das explizite Euler-Verfahren Vorteile im Hinblick auf den Berechnungsaufwand aufzeigte. Das Radau2A-Verfahren und \textit{ODE23s} hingegen liefern selbst bei einer sehr geringen Anzahl von Optimierungsstellen bereits sehr gute Ergebnisse.
\\Der zweite Ansatz verwendet indirekte Lösungsmethoden, d.h. zu Beginn wird über das Pontryaginsche Minimumsprinzip die optimale Steuerung $u^*(t)$ in Abhängigkeit der Zustandsvariable $x(t)$ und der adjungierten Variable $\lambda(t)$ aufgestellt. Anschließend wird mit den gewonnen Informationen ein Randwertproblem abgeleitet, welches über gängige Verfahren lösbar ist. Für die Lösung des Randwertproblems wurden vier verschiedene Verfahren ausprobiert: Das Einfachschieß-, das Mehrfachschieß-, das Differenzen- und das Matlab-Verfahren \textit{bvp4c}. Alle vier Methoden lieferten unter gewissen Voraussetzungen ebenfalls gute Ergebnisse, wiesen jedoch eine starke Abhängigkeit von der Startschätzung auf. Des Weiteren wird in \autoref{fig:CompareBound} ersichtlich, dass die Annahme (\ref{eq:anforderung_lambda2<lambda1_t>0}), welche $\lambda_1 > \lambda_2$  fordert, gerechtfertigt ist. Diese stellt die Regularität der Hamilton-Funktion und somit die zulässige Anwendung des Pontryaginschen Minimumsprinzips sicher.
\\Zusammenfassend lässt sich sagen, dass sowohl die direkten, als auch die indirekten Lösungsmethoden sinnvolle Ergebnisse berechnen. Der Reaktant $A_1$ bzw. $x_1(t)$ wird kontinuierlich abgebaut, wohingegen die Konzentration des Produktes $x_2(t)$ gleichmäßig steigt. Am Ende zeichnet sich ein Gleichgewicht ab. Die optimale Steuerung $u^*(t)$ weist keine Sprünge auf. Die Analyse der Steifheit und die sinnvollen Ergebnisse des expliziten Euler-Verfahrens implizieren ein nicht steifes Problem, womit die Verwendung verschiedenster Verfahren gerechtfertigt ist.
